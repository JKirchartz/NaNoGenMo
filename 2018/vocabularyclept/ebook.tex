mage%%%%%%%%%%%%%%%%%%%%%%%%%%%%%%%%%%%%%%%%%
% eBook 
% LaTeX Template
% Version 1.0 (29/12/14)
%
% This template has been downloaded from:
% http://www.LaTeXTemplates.com
%
% Original author:
% Luis Cobo (luiscobogutierrez@gmail.com) with extensive modifications by:
% Vel (vel@latextemplates.com)
%
% Adapted for use and inclusion in https://tildegit.org/biglysmalls/TildePublishingUnlimited by:
% biglysmalls (@ a tilde near you!)
%
% License:
% CC BY-NC-SA 3.0 (http://creativecommons.org/licenses/by-nc-sa/3.0/)
%
%%%%%%%%%%%%%%%%%%%%%%%%%%%%%%%%%%%%%%%%%

%----------------------------------------------------------------------------------------
%	DOCUMENT CONFIGURATIONS AND INFORMATION
%----------------------------------------------------------------------------------------

\documentclass[oneside,11pt]{memoir} % Font size

\input{structure} % Include the file that specifies the document structure and layout
\usepackage{lipsum} % for use of the \lipsum macro command to generate dummy text in the chapters.

\title{Compelling Title} % Book title
\author{Ghost Writer, Esq.} % Author
\newcommand{\edition}{Second Edition} % Book edition

%----------------------------------------------------------------------------------------

\begin{document}

%----------------------------------------------------------------------------------------
%	TITLE PAGE
%----------------------------------------------------------------------------------------

\thispagestyle{empty} % Suppress page numbering
%\ThisCenterWallPaper{1.05}{cover/doodoobutt.jpg} % Add the background image, the first argument is the scaling - adjust this as necessary so the image fits the entire page
\ThisCenterWallPaper{1.12}{cover/littlered.jpg} % Add the background image, the first argument is the scaling - adjust this as necessary so the image fits the entire page
% You can supply your own .jpg by putting it in the 'cover' directory and change the filename above to reflect your cover image.
% Or you can remove it by deleting or commenting out the command above (insert % at the begining of the line)

% This tikzpicture code creates the overlay box on the cover page and populates it with author/title/edition info from the document information section above.
\begin{tikzpicture}[remember picture,overlay]
\node [rectangle, rounded corners, fill=white, opacity=0.75, anchor=south west, minimum width=3cm, minimum height=6cm] (box) at (-0.5,-10) (box){}; % White rectangle - "minimum width/height" adjust the width and height of the box; "(-0.5,-10)" adjusts the position on the page
\node[anchor=west, color01, xshift=-1.5cm, yshift=-0.4cm, text width=2.9cm, font=\sffamily\scriptsize] at (box.north){\edition}; % "Text width" adjusts the wrapping width, "xshift/yshift" adjust the position relative to the white rectangle
\node[anchor=west, color01, xshift=-1.5cm, yshift=-2cm, text width=2.9cm, font=\sffamily\bfseries\scshape\Large] at (box.north){\thetitle}; % "Text width" adjusts the wrapping width, "xshift/yshift" adjust the position relative to the white rectangle
\node[anchor=west, color01, xshift=-1.5cm, yshift=-5cm, text width=2.9cm, font=\sffamily\bfseries] at (box.north){\theauthor}; % "Text width" adjusts the wrapping width, "xshift/yshift" adjust the position relative to the white rectangle
\end{tikzpicture} 
\newpage % Make sure the following content is on a new page

%----------------------------------------------------------------------------------------
%	COPYRIGHT PAGE
%----------------------------------------------------------------------------------------

\newpage
~\vfill
\thispagestyle{empty}

\noindent Copyright \copyright\ 2019 John Smith\\ % Copyright notice

\noindent \textsc{Published by Tilde Publishing Unlimited}\\ % Publisher

\noindent \textsc{http://tilde.town}\\ % URL

\begin{tiny}
\noindent Licensed under the Creative Commons Attribution-NonCommercial 3.0 Unported License (the ``License''). You may not use this file except in compliance with the License. You may obtain a copy of the License at \url{http://creativecommons.org/licenses/by-nc/3.0}. Unless required by applicable law or agreed to in writing, software distributed under the License is distributed on an \textsc{``as is'' basis, without warranties or conditions of any kind}, either express or implied. See the License for the specific language governing permissions and limitations under the License.\\ % License information, replace this with your own license (if any)
\end{tiny}

\noindent \textit{First printing, March 2019} % Printing/edition date


%----------------------------------------------------------------------------------------
%	TABLE OF CONTENTS
%----------------------------------------------------------------------------------------

\tableofcontents % Prints the table of contents

%----------------------------------------------------------------------------------------
%	INTRODUCTION SECTION
%----------------------------------------------------------------------------------------

\chapter*{Introduction/Prologue} % Introduction chapter suppressed from the table of contents
% begin prologue
\begin{quote}
This is one of my finer quotations.\\
--John Smith
\end{quote}

This is a great place to write an introduction or prologue\footnote{You can even use a footnote to seem smarter}.
% end prologue
%You can move the intro/prologue to a separate file like with the subsequent chapters. To try it, comment out
% everything between the lines % begin prologue .. % end prologue and uncomment the following line:
%\include{inputs/prologue.md}
% The prologue.md file just contains those same six lines from above.

% test of being able to insert a chapter and book part before the main body of the work:
% we can do so by defining a new part, defining a new chapter, and including a body of text
% the compilation process will take care of the rest. Uncomment the following 3 lines to test it out.
%\part{The Interruption}
%\chapter{An Interruption in Parts and Chapters}
%\include{inputs/firstchap.md}

%----------------------------------------------------------------------------------------
%	BOOK PART
%----------------------------------------------------------------------------------------

\part{The Word Salads}

%----------------------------------------------------------------------------------------
%	CHAPTER ONE
%----------------------------------------------------------------------------------------

\chapter{Little Red Riding Hood}
\include{inputs/lrrh.md}

%----------------------------------------------------------------------------------------
%	CHAPTER TWO
%----------------------------------------------------------------------------------------

\chapter{Hansel and Gretel}
\include{inputs/hng.md}

%----------------------------------------------------------------------------------------
%	CHAPTER THREE
%----------------------------------------------------------------------------------------

\chapter{More Lipsum}
\include{inputs/thirdchap.md}

%----------------------------------------------------------------------------------------
%	BOOK PART
%----------------------------------------------------------------------------------------

\part{The Verses}

%----------------------------------------------------------------------------------------
%	SONG ONE
%----------------------------------------------------------------------------------------

% Test to demonstrate how to include poetry or lyrics into this kind of ebook
% Here it is just inserted before the first chapter
% Song and lyrics generated by www.song-lyrics-generator.org.uk
% using random inputs, in the style of Taylor Swift
% linebreaks in inputs/song.md produced with manual doublespacing.
% Some more options for working with poems and verses here: https://stackoverflow.com/questions/1236580/latex-memoir-the-verse-environment-and-poemtitle#1236727
\chapter{A song}
\begin{quote} % Using a quote block to insert a by-line for the verse. Remove/comment out if unnecessary (like in the next set of verses)
    This Love is Super But It's Poor\\ % the \\ at the end of the line forces a linebreak before the by line:
    by: skynet
\end{quote}
\settowidth{\versewidth}{This Love is Super But It's Poor} % The width of the verse lines is set by the text within the braces. Note the different width in the next song.
\begin{verse}[\versewidth]
\ldots \\*
\include{inputs/song.md}
\end{verse}

\chapter{This Love is Super But It's Poor} %Within the braces as this chapter name is the title of the poem/song/piece.
\settowidth{\versewidth}{A song, a song!}
\begin{verse}[\versewidth]
\ldots \\*
\include{inputs/song.md}
\end{verse}

\end{document}
